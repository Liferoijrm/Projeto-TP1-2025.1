\documentclass[12pt]{article}

\usepackage{sbc-template}
\usepackage[brazil,american]{babel}
\usepackage[utf8]{inputenc}

\usepackage{graphicx}
\usepackage{url}
\usepackage{float}
\usepackage{listings}
\usepackage{color}
\setlength{\marginparwidth}{2cm}
\usepackage{todonotes}
\usepackage{algorithmic}
\usepackage{algorithm}
\usepackage{hyperref}

\sloppy


\title{Trabalho Prático\\ 
Cartolitos CF}

%author{Nome do Aluno, Matrícula\\
\author{Élvis Júnior, 24/1038700\\
        Gustavo Alves, 24/1020779\\
        Pedro Marcinoni, 24/1002396\\
        Grupo 1
}

\address{Dep. Ciência da Computação -- Universidade de Brasília (UnB)\\
  CIC0197 - Técnicas de Programação I\\
  \email{elvismirandajr@gmailcom, gusfring.a@gmail.com, pedroextrer@gmail.com}
}

\begin{document}
\maketitle

\selectlanguage{american}
\begin{abstract}
  This document describes the final project of the course "Técnicas de Programação I" at Universidade de Brasília. The project is a desktop application that allows users to manage their fantasy football teams, providing features such as team creation, player selection, and match simulation. The application is built following object-oriented programming principles and utilizes design patterns to ensure maintainability and scalability.
\end{abstract}
\selectlanguage{brazil}

\begin{resumo}
  Este documento descreve o projeto final da disciplina Técnicas de Programação I da Universidade de Brasília. O projeto é uma aplicação desktop que permite aos usuários gerenciar suas equipes de futebol \textit{fantasy}, oferecendo recursos como criação de equipes, seleção de jogadores e simulação de partidas. A aplicação é construída seguindo princípios de programação orientada a objetos e utiliza padrões de design para garantir manutenibilidade e escalabilidade.
\end{resumo}


\section{Descrição do Problema}
\label{sec:descricao}

Como bons brasileiros, quem não gosta de futebol? Foi pensando nisso que a emissora mais famosa do país, a Rede Globe, criou a Cartolitos CF, o novo Fantasy Game do momento, que reflete em tempo real os resultados estatísticos da Copa do Mundo de Clubes, o mais novo fenômeno do futebol. Veja também \cite{wiki:cartola,wiki:fantasy}.

Disponível para computador, o aplicativo permite que os usuários escalem seus times comprando jogadores dos 32 clubes do torneio com “cartoletas” — a moeda virtual do jogo — e pontue de acordo com o desempenho real desses atletas em campo na rodada. Assim, os participantes podem comparar seus resultados com os amigos em uma liga estilo tiro-curto, na qual se considera apenas uma única rodada.

Para colocar esse projeto em prática, a empresa designou três estudantes de Ciência da Computação — Elvis Miranda, Gustavo Alves e Pedro Marcinoni — para desenvolver um programa capaz de auxiliar na logística de gestão de qualquer liga de tiro-curto, servindo depois como base para o funcionamento integral do aplicativo.


\section{Definição das regras de negócio}
\label{sec:regras}

Descrever aqui as regras de negócio do projeto. O que é necessário para o funcionamento do sistema? Quais as entradas e saídas? Quais os principais componentes do sistema?

\section{Diagrama de classes final}
\label{sec:classes}

Colocar aqui o diagrama de classes final do projeto. O diagrama deve ser gerado a partir do código fonte do projeto. O diagrama deve ser legível e conter todas as classes do projeto.

\section{Telas}
\label{sec:telas}

Colocar aqui as telas do projeto. As telas devem ser geradas a partir do código fonte do projeto. As telas devem ser legíveis e conter todas as funcionalidades do projeto.

\section{Conclusão}
\label{sec:conclusao}

Colocar aqui a conclusão do projeto. O que foi aprendido? O que poderia ser melhorado? Quais as dificuldades encontradas? Quais os próximos passos?

\bibliographystyle{sbc}
\bibliography{relatorio}  %Aqui é a definição do arquivo .bib a ser usado pelas referências


\end{document}
